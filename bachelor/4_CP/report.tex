\documentclass[dvipsnames, pdf, unicode, 12pt, a4paper, oneside, fleqn]{article}
\usepackage[utf8]{inputenc}
\usepackage[T2B]{fontenc}
\usepackage[english,russian]{babel}


\usepackage{listings}
\usepackage{longtable}
\oddsidemargin=-0.4mm
\textwidth=160mm
\topmargin=4.6mm
\textheight=210mm
\usepackage{geometry}
%% Страницы диссертациия должны иметь следующие поля:
%% левое --- 25 мм, правое --- 10 мм, верхнее --- 20 мм, нижнее --- 20 мм.
%% Абзацный отступ должен быть одинаковым по всему тексту и равен пяти знакам.
\geometry{
 a4paper,
 total={170mm,257mm},
 right=10mm,
 left=10mm,
 top=20mm,
 bottom=20mm,
}
\pagenumbering{gobble}

\usepackage{multicol}
\usepackage[]{amsmath}
\usepackage{multirow}


% THIS IS MY NEWLY DEFINED COMMAND
\newcommand\tline[2]{$\underset{\text{#1}}{\text{\underline{\hspace{#2}}}}$}

\usepackage{csquotes}
\DeclareQuoteStyle{russian}
    {\guillemotleft}{\guillemotright}[0.025em]
    {\quotedblbase}{\textquotedblleft}
\ExecuteQuoteOptions{style=russian}

\usepackage{longtable,array}

\newcolumntype{C}[1]{>{\centering\arraybackslash}p{#1}}
\setlength{\extrarowheight}{10pt}

\begin{document}

\begin{titlepage}
\begin{center}
\bfseries{\Large Министерство образования и науки\\Российской Федерации}

\vspace{12pt}

\bfseries{\Large Московский авиационный институт\\ (национальный исследовательский университет)}

\vspace{48pt}


%{\large Факультет информационных технологий и прикладной математики}

\vspace{36pt}


%{\large Кафедра вычислительной математики и~программирования}

\vspace{48pt}

{\huge ЖУРНАЛ}

\vspace{12pt}

{\large ПО ПРОИЗВОДСТВЕННОЙ ПРАКТИКЕ}


\end{center}

\vspace{72pt}

\begin{flushleft}
Наименование практики: {\itshape вычислительная}\\
Студент: А.\,В. Полей-Добронравова \\
Факультет №8, курс 2, группа 7 \\
\end{flushleft}

\vspace{12pt}

\begin{flushleft}
Практика с 29.06.20 по 12.07.20
\end{flushleft}

\vfill

\begin{center}
\bfseries Москва, \the\year
\end{center}
\end{titlepage}

\pagebreak

\begin{center}
\bfseries{\large ИНСТРУКЦИЯ }

\vspace{12pt}

\bfseries{о заполнении журнала по производственной практике}
\end{center}

\begin{multicols}{2}
{\small
Журнал по производственной практике студентов имеет единую форму для всех видов практик.

Задание в журнал вписывается руководителем практики от института в первые три-пять дней пребывания студентов на практике в соответствии с тематикой, утверждённой на кафедре до начала практики. Журнал по производственной практике является основным документом для текущего и итогового контроля выполнения заданий, требований инструкции и программы практики.

Табель прохождения практики, задание, а также технический отчёт выполняются каждым студентом самостоятельно.

Журнал заполняется студентом непрерывно в процессе прохождения всей практики и регулярно представляется для просмотра руководителям практики. Все их замечания подлежат немедленному выполнению.

В разделе «Табель прохождения практики» ежедневно должно быть указано, на каких рабочих местах и в качестве кого работал студент. Эти записи проверяются и заверяются цеховыми руководителями практики, в том числе мастерами и бригадирами. График прохождения практики заполняется в соответствии с графиком распределения студентов по рабочим местам практики, утверждённым руководителем предприятия.
В разделе «Рационализаторские предложения» должно быть приведено содержание поданных в цехе рационализаторских предложений со всеми необходимыми расчётами и эскизами. Рационализаторские предложения подаются индивидуально и коллективно.

Выполнение студентом задания по общественно-политической практике заносятся в раздел «Общественно-политическая практика». Выполнение работы по оказанию практической помощи предприятию (участие в выполнении спецзаданий, работа сверхурочно и т.п.) заносятся в раздел журнала «Работа в помощь предприятию» с последующим письменным подтверждением записанной работы соответствующими цеховыми руководителями.
Раздел «Технический отчёт по практике» должен быть заполнен особо тщательно. Записи необходимо делать чернилами в сжатой, но вместе с тем чёткой и ясной форме и технически грамотно. Студент обязан ежедневно подробно излагать содержание работы, выполняемой за каждый день. Содержание этого раздела должно отвечать тем конкретным требованиям, которые предъявляются к техническому отчёту заданием и программой практики. Технический отчёт должен показать умение студента критически оценивать работу данного производственного участка и отразить, в какой степени студент способен применить теоретические знания для решения конкретных производственных задач.

Иллюстративный и другие материалы, использованные студентом в других разделах журнала, в техническом отчёте не должны повторяться, следует ограничиваться лишь ссылкой на него. Участие студентов в производственно-технической конференции, выступление с докладами, рационализаторские предложения и т.п. должны заноситься на свободные страницы журнала.

{\bfseries Примечание.} Синьки, кальки и другие дополнения к журналу могут быть сделаны только с разрешения администрации предприятия и должны подшиваться в конце журнала.

Руководители практики от института обязаны следить за тем, чтобы каждый цеховой руководитель практики перед уходом студентов из данного цеха в другой цех вписывал в журнал студента отзывы об их работе в цехе.

Текущий контроль работы студентов осуществляется руководители практики от института и цеховыми руководителями практики заводов. Все замечания студентам руководители делают в письменном виде на страницах журнала, ставя при этом свою подпись и дату проверки.

Результаты защиты технического отчёта заносятся в протокол и одновременно заносятся в ведомость и зачётную книжку студента.

{\bfseries Примечание.} Нумерация чистых страниц журнала проставляется каждым студентом в своём журнале до начала практики.
}
\end{multicols}

\begin{center}
С инструкцией о заполнении журнала ознакомились:
\end{center}

«\hspace{0.5cm}» \tline{(дата)}{1.5in} \the\year\,г.\hfillСтудент Полей-Добронравова А.\,В. \tline{(подпись)}{1in}
\pagebreak

\begin{center}
\bfseries{\large ЗАДАНИЕ}
\end{center}

кафедры 806 по вычислительной/исследовательской практике:\newline
Игра 3д на Unity, вид от 1 лица, есть как минимум одна локация и взаимодействие с ней.

\vspace*{\fill}
Руководитель практики от института:

\vspace{5pt}
\enquote{\hspace{0.5cm}} \tline{(дата)}{1.5in} \the\year\,г.\hfillКухтичев А.\,A. \tline{(подпись)}{1in}
\pagebreak

\begin{center}
\bfseries{\large ТАБЕЛЬ ПРОХОЖДЕНИЯ ПРАКТИКИ}
\end{center}

\begin{longtable}{|C{2cm}|C{6cm}|C{1.7cm}|C{1.5cm}|C{1.5cm}|C{2.8cm}|}
    \hline
    {\bfseries Дата} & {\bfseries Содержание или наименование проделанной работы} & {\bfseries Место работы} & \multicolumn{2}{c|}{{\bfseries Время работы}} & {\bfseries Подпись цехового руководителя}\\
    \cline {4-5} & & & Начало & Конец & \\
    \endfirsthead
    \hline
    {\bfseries Дата} & {\bfseries Содержание или наименование проделанной работы} & {\bfseries Место работы} & \multicolumn{2}{c|}{{\bfseries Время работы}} & {\bfseries Подпись цехового руководителя}\\
    \cline {4-5} & & & Начало & Конец & \\
    \hline
    \endhead
    \multicolumn{6}{c}{\textit{Продолжение на следующей странице}}
    \endfoot
    \endlastfoot
    \hline
    29.06.2020 & Получение задания & МАИ & 9:00 & 18:00 & \\
    \hline
    01.07.2020 & Разработка концепции компьютерной игры. Установка необходимых IDE: Unity, Blender. Изучение функционала добавления 3д фигур, способности им двигаться под действием законов физики, экспорта игры на ПК. & МАИ & 9:00 & 18:00 & \\
    \hline
    02.07.2020 & Изучение скриптинга на С шарп для Unity. Создание prefab. & МАИ & 9:00 & 18:00 & \\
    \hline
    03.07.2020 & Изучение User Interface на Unity, создание интерактивных кнопок, поля для ввода. Разработка дизайна игры, переключение между сценами.  & МАИ & 9:00 & 18:00 & \\
    \hline
    04.07.2020 & Написание скрипта для перемещения объекта 2 способами: изменяя координату и под действием законов физики. Управление игрой с помощью кнопок клавиатуры. Поворот камеры игрока при движении мышки. & МАИ & 9:00 & 18:00 & \\
    \hline
    05.07.2020 & Основы 3д моделирования в Blender (режимы Blender, создание простейших фигур, экстрадуирование, понятия материала и текстуры) & МАИ & 9:00 & 18:00 & \\
    \hline
    06.07.2020 & Наложение текстур, масштабирование текстур, редактирование текстур. Разница между текстурированием в Blender и Unity. Виды шейдеров в Unity. & МАИ & 9:00 & 18:00 & \\
    \hline
    07.07.2020 & UV-развёртка в Blender, наложение текстур по развёртке. & МАИ & 9:00 & 18:00 & \\
    \hline
    09.07.2020 & Экспорт моделей из Blender, импорт моделей в Unity. Типы освещения в Unity, настройка освещения в локации. Музыкальное сопровождение сцены в Unity. & МАИ & 9:00 & 18:00 & \\
    \hline
    10.07.2020 & Попытка настроить отслеживание столконвений с импортированным в Unity объектом, неуспешно, столкновения обнаруживаются только с объектами, созданными в самом Unity, из-за чего персонаж игры проходит через стены и не взбирается по лестнице.   & МАИ & 9:00 & 18:00 & \\
    \hline
    11.07.2020 & Импорт готовых моделей из интернета и обустраивание основной локации.  & МАИ & 9:00 & 18:00 & \\
    \hline
    12.07.2020 & Сдача журнала & МАИ & 9:00 & 18:00 &  \\
    \hline
\end{longtable}

\pagebreak

\begin{center}
\bfseries{\large Отзывы цеховых руководителей практики}
\end{center}
Студент Полей-Добронравова А.\,В. разработала 3д игру на Unity с одной локацией и работой с ней, научилась основам 3д моделирования в Blender.

Презентация защищена на комиссии кафедры 806. Работа выполнена в полном объёме. Рекомендую на оценку \enquote{\hspace{2cm}}. Все материалы сданы на кафедру.
\pagebreak


\begin{center}
\bfseries{\large ПРОТОКОЛ }

\vspace{12pt}

\bfseries{ЗАЩИТЫ ТЕХНИЧЕСКОГО ОТЧЁТА}
\end{center}
\noindent
по {\itshapeвычислительной практике}

\vspace{8pt}
\noindent
студентами:
\noindent
Полей-Добронравова Амелия Вадимовна

\begin{longtable}{p{7cm}|p{11cm}}
    \hline
    {\bfseries Слушали:} & {\bfseries Постановили:}  \\
    \endfirsthead
    \hline
    {\bfseries Слушали:} & {\bfseries Постановили:}  \\
    \hline
    \endhead
    \multicolumn{2}{c}{\textit{Продолжение на следующей странице}}
    \endfoot
    \endlastfoot
    Отчёт практиканта & считать практику выполненной и защищённой на\\
    \rule{0pt}{425pt} & Общая оценка: \underline{\hspace{2in}}\\
    \rule{0pt}{15pt} & \\
    \hline
\end{longtable}

\vfill

\noindent\begin{tabular}{@{}l l l}
Руководители: & Зайцев В.\,Е. & \underline{\hspace{2in}}\\
 \rule{0pt}{10pt} & Кухтичев А.\,А. & \underline{\hspace{2in}}
\end{tabular}
\vspace{12pt}

\noindent
Дата: 12 июля \the\year\,г.
\pagebreak


\begin{center}
\bfseries{\large МАТЕРИАЛЫ ПО РАЦИОНАЛИЗАТОРСКИМ ПРЕДЛОЖЕНИЯМ}
\end{center}

Улучшить качество моделей, расширить локацию. Добавить анимацию объектам игры, поставить музыку на фон, звуковое сопровождение эффектов анимации (пример: скрип двери при открытии). \\ Перестроить проект таким образом, чтобы наличие объектов и их свойства были прописаны и настроены через скрипт для простоты работы с проектом на другой вычислительной машине. \\ 

\pagebreak

\begin{center}
\bfseries{\large ТЕХНИЧЕСКИЙ ОТЧЁТ ПО ПРАКТИКЕ}
\end{center}

\section*{Архитектура}
Язык С Шарп \\ Assets: 3д Модели, Материалы, Текстуры, Скрипты, Сцены; \\ UnityPackage; \\ ProjectSettings.
\section*{Описание}
Моя программа на данном этапе разработке скорее заготовка под игру. Создан UI - пользовательский интерфейс, локация, игрок может перемещаться по локации с помощью клавиатуры и поворачивать персонажа с помощью компьютерной мыши. С помощью горячих клавиш клавиатуры и интерактивных кнопок UI происходит смена локации.

\section*{Реализация }

При реализации использовалось две IDE - среды разработки: Unity, Blender. Программы-скрипты на языке С шарп. \\
Я пробовала разные способы реализации задач проекта. Поробовала сама поделать 3Д модели в Blender, так же попробовала там наложить на них текстуры. При экспорте из Blender и последующем импорте в Unity столкнулась с проблемой, что материалы и текстуры неправильно подгружаются, поэтому текстурировала в последствии в самом Unity. Так же попробовала взять чужие готовые модели и разместить на своей локации. \\
В работе непосредственно с игровыми моментами в Unity я часть свойств объектов прописывала через скрипты С шарп, часть с помощью самой IDE. Большинство скриптов пишутся напрямую работая с библиотеками Unity, поэтому важно было разобраться с ними. 
\section*{Тестирование}
Пошаговое тестирование представлено на защите руководителю практики. Его сложно отобразить фотографиями, нужен видеорежим. Тестирование программы подразумевает использование всего вышеперечисленного функционала (Работа с UI, Перемещение между сценами, Перемещение внутри локации)

\section*{Ссылка на GitHub \\ https://github.com/tainella/computationalpractice}
\pagebreak

\end{document}
